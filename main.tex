%CS 109 Problem Set Template
%Jackie Ennis
\title{CS109 PSET Template}
\documentclass{article}
	% basic article document class
	% use percent signs to make comments to yourself -- they will not show up.
\usepackage{amsmath}
\usepackage{amssymb}
	% packages that allow mathematical formatting
\usepackage{graphicx}
	% package that allows you to include graphics
    %includegraphic[width=\textwidth]{FILENAME}
\usepackage[top=1in, bottom=1in, left=1in, right=1in]{geometry}
\frenchspacing
	% one space after periods
\usepackage{fancyhdr}
	% allows custom headers
\usepackage{relsize}
\pagestyle{fancy}
\setlength{\headheight}{24pt}% ...at least 23.1004pt
\lhead{16-720AC Computer Vision\\ Homework 2} 
\rhead{David Wong (davidwong@cmu.edu) \\ Andrew ID: DBWONG}
\cfoot{\thepage}
\renewcommand{\footrulewidth}{0.3pt} 
	%footer

\begin{document}
\thispagestyle{fancy} %shows header/footer
\begin{enumerate}
% Problem 1 --------------------------------------------------------------
\item 
	\begin{enumerate}
	
	 \item\textbf{Q1.1}

        Two homogenous coordinate vectors are equivalent if they are proportional to each other:\\
        
        \begin{bmatrix}
        $x_i$ \\
        $y_i$ \\
        $z_i$ \end{bmatrix}\ $\equiv$ $\lambda$ \begin{bmatrix}
        $x_i^'$ \\
        $y_i^'$ \\
        $z_i^'$ \end{bmatrix}, for all non-zero $\lambda$.\\
        
        The linear mapping of homogenous coordinates is as follows: \\
        \begin{bmatrix}
        x \\
        y \\
        z \end{bmatrix}\ = \begin{bmatrix}
        1 & 0 & 0 & 0 \\
        0 & 1 & 0 & 0 \\
        0 & 0 & 1 & 0 \\\end{bmatrix}\ \begin{bmatrix}
        X \\
        Y \\
        Z \\
        1 \end{bmatrix} this represents either  3 x 4 projection matrices P1 or P2.
        
        All points $x_\pi$ lying in plane $\pi$ between the two camera views $x$ and $x^'$ have zero z coordinate. Then the 3 x 4 matrices P1 and P2 reduces to:
        
        \begin{bmatrix}
        x_1 \\
        x_2 \\
        x_3 \end{bmatrix}\ = \begin{bmatrix}
        p_1_1 & p_1_2 & p_1_3 & p_1_4 \\
        p_2_1 & p_3_2 & p_2_3 & p_2_4 \\
        p_3_1 & p_3_2 & p_3_3 & p_3_4 \end{bmatrix} \begin{bmatrix}
        X \\
        Y \\
        0 \\
        1 \end{bmatrix} = \begin{bmatrix}
        p_1_1 & p_1_2 & p_1_4 \\
        p_2_1 & p_3_2 & p_2_4 \\
        p_3_1 & p_3_2 & p_3_4 \end{bmatrix} \begin{bmatrix}
        X \\
        Y \\
        1 \end{bmatrix}\\
        
        which can be rewritten as:\begin{bmatrix}
        x_1 $^'$\\
        x_2 $^'$\\
        x_3 $^'$\end{bmatrix}\ = \begin{bmatrix}
        h_1_1 & h_1_2 & h_1_3 \\
        h_2_1 & h_3_2 & h_2_3 \\
        h_3_1 & h_3_2 & h_3_3 \end{bmatrix} or $x_1^'$ = H$x_2$.\\
        
        \\ For camera matrix M and M$^'$ of cameras C and $C^'$, we see that MX = $(MH)(H^-^1X)$, and similarly for M$\prime$, corresponding to a 3D point X - representing 3D vector \begin{bmatrix} $x_i & y_i & z_i$ \end{bmatrix}$^T$ then they are also matched points to the pair of cameras C and $C^'$ with camera matrices MH and $M^'$H, corresponding to the point $H^-^1$X. Therefore this proves that the homography H exists and satisfies equation 1.\\
        
    %Q1.2 CORRESPONDENCES
    
    \item \textbf{Q1.2}
    
    1. There are 8 degrees of freedom for \textbf{h}.\\
    2. We need at least 4 points, or 2 point pairs, to solve \textbf{h}.\\
    3. \begin{bmatrix}
        x$^'$\\
        y$^'$\\
        1\end{bmatrix}\ $\sim$ \begin{bmatrix}
        h_1_1 & h_1_2 & h_1_3 \\
        h_2_1 & h_3_2 & h_2_3 \\
        h_3_1 & h_3_2 & h_3_3 \end{bmatrix} \begin{bmatrix}
        x\\
        y\\
        1\end{bmatrix}\\
        multiply and expand out constituents for $x^'$ and $y^'$:\\
        $x^'$ = ($h_1_1x$ + $h_1_2y$ + $h_1_3$) / ($h_3_1x$ + $h_3_2y$ + $h_3_3$)\\
        $y^'$ = ($h_2_1x$ + $h_2_2y$ + $h_2_3$) / ($h_3_1x$ + $h_3_2y$ + $h_3_3$)\\
        
        Impose unit vector constraint such that $h_1_1^2 + h_1_2^2 + ... + h_3_2^2 + h_3_3^2 = 1$\\ 
        
        \begin{bmatrix}
        $x_1 & y_1 & 1 & 0 & 0 & 0 & -x_1 x_1^` & -y_1 x_1^'$ \\
        0 & 0 & 0 & $x_1 & y_1 & 1 & -x_1 y_1^` & -y_1 y_1^'$ \\
        $x_2 & y_2 & 1 & 0 & 0 & 0 & -x_2 x_2^` & -y_2 x_2^'$ \\
        0 & 0 & 0 & $x_2 & y_2 & 1 & -x_2 y_2^` & -y_2 y_2^'$ \\ 
        $x_3 & y_3 & 1 & 0 & 0 & 0 & -x_3 x_3^` & -y_3 x_3^'$ \\ 
        0 & 0 & 0 & $x_3 & y_3 & 1 & -x_3 y_3^` & -y_3 y_3^'$ \\
        $x_4 & y_4 & 1 & 0 & 0 & 0 & -x_4 x_4^` & -y_4 x_4^'$ \\
        0 & 0 & 0 & $x_4 & y_4 & 1 & -x_4 y_4^` & -y_4 y_4^'$ \end{bmatrix}\ \begin{bmatrix}
        h_1_1 \\
        h_1_2 \\
        h_1_3 \\
        h_2_1 \\
        h_2_2 \\
        h_2_3 \\
        h_3_1 \\
        h_3_2 \end{bmatrix} = \begin{bmatrix}
        $x_1^'$ \\
        $y_1^'$ \\
        $x_2^'$ \\
        $y_2^'$ \\
        $x_3^'$ \\
        $y_3^'$ \\
        $x_4^'$ \\
        $y_4^'$ \end{bmatrix}\\
       
    4. Computing singular vector corresponding to the least singular value of \textbf{A} to compute \textbf{h}. \\
    Trivial solution for h occurs if we set $h_3_3$ = 1, but the actual value of $h_3_3$ = 0; this would result in the inability to get the right answer. \\
    
    For Ah = 0, with A as an m x n matrix of known values and h as an n-element vector of unknowns, the conditions for a solution to exist for n-element vector h are:\\
    1) rank of A will never exceed n;\\
    2) if rank of A is exactly n, then the only solution is the trivial solution when h = 0;\\
    3) if rank of A is n-1 and m = n-1, then there exists a unique solution for h up to a scale factor. We have 8 equations for 9 unknowns in h; m = 8, n = 9, and rank(A) = 8;\\
    4) if rank A = n-1 and m $greq$ n, we have an overdetermined system of equations, and would use the linear least squares minimization to arrive at a soluion;\\
    
    Eigenvalues and eigenvectors: \\
    The solution of the linear least squares minimization for Ah is given by the eigengector of $A^T$A\\ which corresponds to its smallest eigenvalue. $A_m_x_n$ = $U_m_x_n$ . $D_n_x_n$ . $V_n_x_n^T$; the column vectors of V are the eigenvectors of $A^T$A.\\
    
    \item \textbf{Q1.3}\\ A camera that is rotating has no translation maps points in 3D as: $\textbf{X}_2$ = R$\textbf{X}_1$, as a special case of homography:\\
    H = R + \(\frac{1}{d}\) T$N^T$, with T = 0\\
    resulting in \widehat{x_2} H $x_1$ = 0 which becomes \widehat{x_2} R $x_1$ = 0.\\
    
    \item \textbf{Q1.4}\\
     \begin{bmatrix}
        u \\
        v \end{bmatrix} = \begin{bmatrix}
        fx/z \\
        fy/z \end{bmatrix}\\ 
    
    \begin{bmatrix}
        u \\
        v \\
        1 \end{bmatrix} $\equiv$
        \begin{bmatrix}
        fx \\
        fy \\
        fz \end{bmatrix} = 
        \begin{bmatrix}
        f & 0 & 0 & 0 \\
        0 & f & 0 & 0 \\
        0 & 0 & f & 0 \\
        0 & 0 & 0 & f\end{bmatrix}  \begin{bmatrix}
        x \\
        y \\
        z \\
        1\end{bmatrix}\\
        For rotation about its center \textbf{C} by an angle $\theta$, there is no translation in Z axis and no skew in x and y axes: \\
        
        \begin{bmatrix}
        $u_0$ \\
        $v_0$ \\
        1 \end{bmatrix} = \begin{bmatrix}
       K | 0\end{bmatrix}  \begin{bmatrix}
       R & 0 \\
       0 & 1 \end{bmatrix}  \begin{bmatrix}
       I & 0\\
       0 & 1\end{bmatrix} = \textbf{$x_1$} \equiv \textbf{H$x_2$}. Therefore for rotation by 2$\theta$, the homography required is \textbf{H^2}.\\
    
    \item \textbf{Q1.5}\\
    Linear estimation of projective transformation parameters from point correspondences in planar homography suffers from poor conditioning of the matrices involved, resulting in a solution that is sensitive to noise in the points.\\
    
    \item \textbf{Q1.6}\\
    Two homogenous coordinate vectors are equivalent if they are propotional to each other, both in 2D and 3D. \\
    \begin{bmatrix}
        u \\
        v \\
        w \\
        t \end{bmatrix} $\equiv$ $\lambda$ \begin{bmatrix}
        $u^'$ \\
        $v^'$ \\
        $w^'$ \\
        $t^'$ \end{bmatrix}, for all non-zero $\lambda$.\\
        
        If t = 0, the physical line comprising multiple points are defined at infinity (infinitely far away). These get mapped onto the image as projected lines converging into a vanishing point. Therefore the project \textbf{P} in \textbf{x} = \textbf{PX} preserves lines.
     
    
    
  	\end{enumerate}
% Problem 2 --------------------------------------------------------------
	\newpage
	\item 
	\begin{enumerate}
		\item \textbf{Q2.1.1}  \\
        The HARRIS corner detector detects more number of points than others at the expense of computation time.\\ FAST detects the least number of points but requires relatively shorter time for matching features.\\
        
        Ref: Verma, S.B., Chandran, S. (2016) Comparative Study of FAST, MSER and Harris for Palmprint Verification System. \textit{Scientific \& Engineering Research}, 7(12).\\      
        
		\item \textbf{Q2.1.2} \\
        The BRIEF descriptor converts images patches into a binary feature vector to represent an object; the binary feature descriptor only contains 1s and 0s, in a vector which is of a type string of 128 to 512 bits.\\
        This is unlike the previous filter banks which seek to represent an image in a matrix of dimensions H x W x Channel, with pixel represented by float values corresponding with signal intensity.\\
        
        \item \textbf{Q2.1.3}\\
        Hamming distance and 
        Nearest Neighbor used to match interest points with BRIEF descriptors.\\
        
        Hamming distance:\\
        Uses XOR incructions on bit sets to determine a match by comparing every descriptor in the first image with descriptors in the second image. Hamming distance calculates the difference (percentage) in gradient intensity arrays between two images. Based on a preset threshold, image comparisons that meet this hamming distance thresholds are considered a match.\\
        
        Nearest Neighbor(NN):\\
        Image features are detected by FAST and then the corner orientations are determined. Meanwhile, corresponding binary descriptors with BRIEF are obtained. Finally, feature points are matched by finding the nearest Euclidean distances. 
        
        NN is simple to implement, flexible to feature/dsistance choices. Howver, the computation cost is high as it computes the distance of each query instance to all training samples. Lazy learner because it doesn't' learn a discriminatory function from training data but memorizes the training data set instead.\\
        
        \textit{Hamming distance vs Euclidean distance}\\
        Hamming distance measures the similiary between two strings of the same length. The Hamming distance between two strings of the same length is the number of positions where corresponding numbers are different. Hamming distance measure how many attributes in the comparison image must be changed in order to match the original image. By comparing attributes of points of interest, we are better able to use this to match image features that have undergone homography.
        
        Euclidean distance measures the shortest distance between two points, which is useful for calculating the distance between two data points in a plane. For image matching, this is less useful as image transformations may make distance measurement between datapoints meaningless for matching.
        
        
        Ref: Zou, J. (2015) A Nearest Neighbor Search Method for Image Matching Based on ORB. \textit{Information and Computational Science}. 12(7): 2691-2700. DOI: 10.12733/jics20105830\\
        
        \newpage
        \item \textbf{Q2.1.4}\\
        Original sigma = 0.15; ratio = 0.7\\
        \includegraphics[scale = 0.55]{Q2.1.4 Output.png}
        
       
        
        \item \textbf{Q2.1.5}\\
        Sigma = 0.6, Ratio = 0.7: no matches were found. \\
        Sigma = 0.15, Ratio = 0.6: found fewer matches vs the original sigma and ratio.\\
        \includegraphics[scale = 0.6]{Q2.1.4 Output_sigma0.15_ratio0.6.PNG}\\
        Sigma = 0.1, Ratio = 0.7: increased matches found especially in graphic region depicting animals\\
        \includegraphics[scale = 0.53]{Q2.1.4 Output_sigma0.1_ratio0.7.PNG}\\
        \newpage
        Sigma = 0.3, Ratio = 0.7: increasing sigma yielded fewer matche, mainly in text region although not all were accurate.\\
        \includegraphics[scale = 0.5]{Q2.1.4 Output_sigma0.3_ratio0.7.PNG}\\
        
        Sigma = 0.3, Ratio = 0.9: increasing ratio yielded greater matches found , more matches found in text region than graphic region depicting animals.\\
        \includegraphics[scale = 0.48]{Q2.1.4 Output_sigma0.3_ratio0.9.PNG}\\
        
        The sigma value set threshold that the contiguous pixels around a pixel \textbf{p} Pixel \textbf{p} is a corner if there exists a set of n contiguous pixels in the circle (of 16 pixels) which are all brighter than $I_p$ + t, or all darker than $I_p$ − t. Increasing the sigma value translates to corner detection involving larger contrast (e.g. such as black text against a white background as seen in the last picture, while a lower contrast section of the image yielded fewer matches. \\
        
        The ratio paramter sets the maximum ratio of distances between first and second closest descriptor in the second set of descriptors. This threshold filters ambiguous matches between the two descriptor sets. The choice of this value depends on the statistics of the chosen descriptor.\\
        
        Ref:\\ $$https:opencv-python-tutroals.readthedocs.io/en/latest/py_tutorials/py_feature2d/py_fast/py_fast.html$$
        
        D.G. Lowe (2004) "Distinctive Image Features from Scale-Invariant Keypoints",
        International Journal of Computer Vision.
        
        \newpage
        \item \textbf{Q2.1.6}\\
        Matches at 50-degree rotation: 37 matches\\
        \includegraphics[scale = 0.25]{Q2.1.6 50deg_rotation_37matches.png}
        
        Matches at 100-degree rotation: 30 matches\\
        \includegraphics[scale = 0.25]{Q2.1.6 100deg_rotation_30matches.png}
        
        Matches at 200-degree rotation: 49 matches\\
        \includegraphics[scale = 0.25]{Q2.1.6 200deg_rotation_49matches.png}
        
        visualize histogram and feature matching result at 3 different orientations. \\
        
        Explain why BRIEF descriptor behaves this way: BRIEF relies on a relatively small number of intensity difference tests to represent an image patch as a binary string. This results in faster construction and matching.\\
        
        Combined histogram of frequency of matches\\
        \includegraphics[scale = 0.2]{Q2.1.6_hist_all.png}\\
        
        Histogram at 10 degrees rotation\\
        \includegraphics[scale = 0.25]{Q2.1.6_hist_10deg.png}\\
        
        Histogram at 50 degrees rotation\\
        \includegraphics[scale = 0.4]{Q2.1.6_hist_50deg.png}\\
        
        Histogram at 60 degrees rotation\\
        \includegraphics[scale = 0.4]{Q2.1.6_hist_60deg.png}\\
        
        
    \item \textbf{Q2.1.x}\\
    
        
        
	\end{enumerate}
\end{enumerate}
\newpage
\end{document}